% Metódy inžinierskej práce

\documentclass[10pt,twoside,slovak,a4paper]{article}

\usepackage[slovak]{babel}
%\usepackage[T1]{fontenc}
\usepackage[IL2]{fontenc} % lepšia sadzba písmena Ľ než v T1
\usepackage[utf8]{inputenc}
\usepackage{graphicx}
\usepackage{url} % príkaz \url na formátovanie URL
\usepackage{hyperref} % odkazy v texte budú aktívne (pri niektorých triedach dokumentov spôsobuje posun textu)

\usepackage{cite}
%\usepackage{times}

\pagestyle{headings}

\title{Využite strojového učenia v softvérovom inžinierstve\thanks{Semestrálny projekt v predmete Metódy inžinierskej práce, ak. rok 2021/22, vedenie:Meno priezvisko}} % meno a priezvisko vyučujúceho na cvičeniach

\author{Adam Partl\\[2pt]
	{\small Slovenská technická univerzita v Bratislave}\\
	{\small Fakulta informatiky a informačných technológií}\\
	{\small \texttt{xpartla@stuba.sk}}
	}

\date{\small 24. Október 2021} % upravte



\begin{document}

\maketitle

\begin{abstract}
V mojom článku sa plánujem zamerať na spôsob využívania strojového učenia v softvérovom 
inžinierstve. V článku môžem popísať akým spôsobom bolo strojové učenie doposiaľ úspešne 
aplikované v rôznych disciplínach softvérového inžinierstva, ako napríklad pri analyzovaní dát, 
generovaní programov, detekcii chýb, rozpoznávaní vzorov alebo ako bolo využité na testovanie 
softvéru, vypočítavanie softvérových požiadaviek a konfiguráciu softvéru. Ďalej sa môžem zaoberať 
otázkami ako, prečo je strojové učenie v softvérom inžinierstve prospešné, v akých rozličných sférach 
sa dá použiť alebo ako dokáže ovplyvniť bežného človeka.
\end{abstract}



\section{Úvod}

Motivujte čitateľa a vysvetlite, o čom píšete. Úvod sa väčšinou nedelí na časti.

Uveďte explicitne štruktúru článku. Tu je nejaký príklad.
Základný problém, ktorý bol naznačený v úvode, je podrobnejšie vysvetlený v časti~\ref{nejaka}.
Dôležité súvislosti sú uvedené v častiach~\ref{dolezita} a~\ref{dolezitejsia}.
Záverečné poznámky prináša časť~\ref{zaver}.



\section{Význam strojového učenia v softvétovom inžinierstve} \label{vyznam}

Z obr.~\ref{f:rozhod} je všetko jasné. 


Aj text môže byť prezentovaný ako obrázok. Stane sa z neho označný plávajúci objekt. Po vytvorení diagramu zrušte znak pred príkazom označte tento riadok ako komentár (tiež pomocou znaku).




\section{Aplikácia SU v softvérovom inžinierstve} \label{aplikacia}

Základným problémom je teda\ldots{} Najprv sa pozrieme na nejaké vysvetlenie (časť~\ref{ina:nejake}), a potom na ešte nejaké (časť~\ref{ina:nejake}).\footnote{Niekedy môžete potrebovať aj poznámku pod čiarou.}

Môže sa zdať, že problém vlastne nejestvuje\cite{Coplien:MPD}, ale bolo dokázané, že to tak nie je~\cite{Czarnecki:Staged, Czarnecki:Progress}. Napriek tomu, aj dnes na webe narazíme na všelijaké pochybné názory\cite{PLP-Framework}. Dôležité veci možno \emph{zdôrazniť kurzívou}.


\subsection{Tvorenie softvéru} \label{ina:Tvorenie}

\paragraph{Veľmi dôležitá poznámka.}
Niekedy je potrebné nadpisom označiť odsek. Text pokračuje hneď za nadpisom.


\subsection{Udržiavanie softvéru} \label{ina:Udrziavanie}

\paragraph{Veľmi dôležitá poznámka.}
Niekedy je potrebné nadpisom označiť odsek. Text pokračuje hneď za nadpisom.

\subsection{Testovanie softvéru} \label{ina:Testovanie}

\paragraph{Veľmi dôležitá poznámka.}
Niekedy je potrebné nadpisom označiť odsek. Text pokračuje hneď za nadpisom.

\subsection{Konfigurácia softvéru} \label{ina:Konfiguracia}

\paragraph{Veľmi dôležitá poznámka.}
Niekedy je potrebné nadpisom označiť odsek. Text pokračuje hneď za nadpisom.

\subsection{Analyzovanie dát} \label{ina:Analyzovanie}

\paragraph{Veľmi dôležitá poznámka.}
Niekedy je potrebné nadpisom označiť odsek. Text pokračuje hneď za nadpisom.



\section{Uplatnenie v reálnom svete} \label{uplatnenie}




\section{Vplyv na používateľa} \label{vplyv}




\section{Záver} \label{zaver} % prípadne iný variant názvu
sadffsdfsdfs
sdfsdfssdfss sdf



%\acknowledgement{Ak niekomu chcete poďakovať\ldots}


% týmto sa generuje zoznam literatúry z obsahu súboru literatura.bib podľa toho, na čo sa v článku odkazujete
\bibliography{literatura}
\bibliographystyle{plain} % prípadne alpha, abbrv alebo hociktorý iný
\end{document}
